\documentclass{./../class/UIR}

% БИБЛИОГРАФИЯ
\bibliographystyle{utf8gost71s}
% Свойства для титульной страницы
\Title          {Добавление новых классов доступа для защищенной операционной
системы}
\StudentGroup   {K8-361}
\StudentName    {{Холявин В.Б.}}
\SupervisorPost {Доцент, кандидат технических наук}
\SupervisorName {Муравьев Сергей Константинович}

\begin{document}

% Титульная страница
\maketitle
% Аннотация
\begin{abstract}
Пояснительная записка к учебно-исследовательской работе:  страниц,  рисунков,  таблиц, список литературы из  наименований.

Ключевые слова: \textit{Linux, SELinux, защищенная операционная система}.

\end{abstract}

% Содержание
\tableofcontents
\StructureElement{Введение}
Модуль безопастности SELinux для операционных систем симейства Linux добавляет к
классической дискреционной системе контроля доступа систему мандатного
контроля доступа. Оставаясь в рамках дискреционной системы контроля доступа, ОС
имеет фундаментальное ограничение в плане разделения доступа процессов к ресурсам
 — доступ к ресурсам основывается на правах доступа пользователя.
 
SELinux позволяет ограничивать доступ работающих приложений только к
тем ресурсам, которые им необходимы. Таким образом, если в
приложении найдется уязвимость, злоумышленник не сможет получить
доступ ни к каким ресурсам, за исключением тех которые приложение
использовало для своей работы.

Существующая политика SELinux включает в себя
множество классов, представляющих системные ресурсы системы (такие как файл или
сокет). Однако элементов графического интерфейса пользователя нет среди классов
доступа SELinux. Добавление подобных элементов может позволить использовать
систему контроля доступа для разграничения возможностей пользователя при
испоьзовании графических приложений. Например, пользователь с соотвествующими
правами, прописанными в модуле политики SELinux сможет воспользоваться кнопкой
или текстовым полем ввода, а пользователь без соотвествующих разрешений - нет. 

В процессе работы был изучен механизм добавления новых классов доступа в
существующую версию политики SELinux. А так же политика, позволяющая управлять
доступом к объектам данного класса в зависимости от прав процесса, получающего
доступ.
 
 
\section{Структура политики SELinux}	
	\subsection{Reference политика}
	Изначально разработкой SELinux занималось Агенство национальной безопастности
	США, затем его исходные коды представлены для свободного скачивания. Компания
	Tresys продолжила разработку политик SELinux. Результатом ее работы стала
	Reference политика. Проект reference политики является в настоящее время
	наиболее популярным, на его основе разрабатываются политики под наиболее
	популярные операционные системы семества Linux.
	 
	Основной целью проекта reference политики является попытка перестроить
	оригинальную политику и превратить ее в более легкую для использования,
	понимания и поддержки. В основе проекта reference политики лежат строгие
	принципы разработки, основанные на понятных принципах проектирования
	програмного обеспечения. При этом reference политика сохраняет все наработки
	оригинальной политики, которые были получены при использовании ее в работе.
	Иными словами ее цель сохранить хорошее и исправить плохое.
	
	Главным недостатком предыдущей версии политики является отсутствие разбиения
	политики на модули и тесная связь исходного кода модуля с финальной политикой.
	Несмотря на то, что макросы добавили абстракции в оригинальную политику, все
	идентификаторы политики (типы, роли, атрибуты и т.д.) являлись в
	действительности глобальными. Изменение одного модуля политики могло
	потребовать знание о многих других модулях. Взаимосвязь между модулями была
	очень тесная и плохо документированная. Создание нового модуля политики
	требовало детального понимания других модулей политики.
	
	Вот некоторые ключевые характеристики reference политики, которые сделали
	разработку политики более простой:
	
	\begin{itemize}
	  \item Одно дерево исходных кодов, которое поддерживает strict и targeted
	  политики, и опционально многоуровневая или многокатегорная защита (MLS, MCS).
	   Политика состоит из одного файла, называемого монолитной политикой. Новая 
	   система загрузки модулей.
	  \item Применение строгих принципов проектирования, заключающихся в
	   основном в слабой связи модулей, с четко определенными интерфейсами и не 
	   глобальное использование типов и других идентификторов. (Так, например, все
	   изменения, сделанные с типом, остаются в рамках одного модуля.)
	  \item Встроенная поддержка генерирования документации, захватывающая 
	  описания интерфейсов модулей. Так что разработчик модуля политики может 
	  использовать интерфейс без понимания того, как интерфейс реализован в модуле.
	\end{itemize}
	
	Помимо упрощения разработки политики, reference политика облегчает проверку
	свойств безопасности, например, для обеспечения сертификации, а так же для 
	расширения поддержки развития инструментов, таких как интегрированная среда 
	разработки и сложные отладчики политики.
	
	
	Больше информации о reference политике, а так же последние ее исходные коды
	можно получить на web сайте проекта http://oss.tresys.com/projects/refpolicy. 
	В последних версиях Fedora core targeted политика основана на reference политике. 
	
\subsection{Описание структуры исходных кодов политики}
	
	Структура reference политики отличается от оригинальной политики. Перед тем,
	как описывать основные детали реализации ссылочной политики, давайте рассмотрим 
	расположение основных файлов и исходных кодах политики, для того что бы 
	ознакомиться с ее файловой структурой.
	
	\subsubsection{Файлы для компиляции и вспомогательные файлы}
	
Следующие файлы и директории используются для сборки или являются
вспомогательными для reference политики:
	\begin{itemize}
\item \begin{verbatim}build.conf\end{verbatim} – Данный файл определяет ряд
параметров сборки которые мы можем изменять.  Данный файл включается в Makefile
в процессе make. Некоторые из опций данного файлу будут рассмотрены ниже.
\item \begin{verbatim}Rules.modula\end{verbatim}r – Данный файл содержит правила
для сборки политики, которая поддерживает загрузку модулей. Он поддерживает
сборку обоих базовых модулей политики и загружаемых модулей политики. Какие
именно модули собирать как часть базового модуля, а какие как загружаемые
определено в файле policy/modules.conf. Параметр сборки MONOLITHIC в build.conf
контролирует модульная или монолитная политика будет собираться.
\item \begin{verbatim}Rules.monolithic\end{verbatim} Если собирается монолитная
политики, данный файл (вместо Rules.modular) включается в Makefile для
определения сборки монолитной политики.
\item \begin{verbatim}config/\end{verbatim} - Эта директория содержит
поддиректории с конфигурационными файлами для каждого варианта политики, которые
мы может построить с reference политикой. Эти конфигурационные файлы обычно те
же самые что и файлы в
\item \begin{verbatim}appconfig/\end{verbatim} директории в example политике.
Эти файлы устанавливаются в директорию с политикой (/etc/selinux/refpolicy) для
поддержки сервисов и приложений.
\item \begin{verbatim}doc/\end{verbatim} эта директория содержит файлы, которые
поддерживают встроенную генерацию документации, которая является частью
reference политики.  Конечную документацию можно посмотреть на web сайте
http://oss.tresys.com/projects/refpolicy/wiki/Documentation. Или выполнив make
html и заглянув в директорию
\item \begin{verbatim}doc/html/support/\end{verbatim} - Эта директория содержит
исходные коды и скрипты для вспомогательных утилит, использующихся в просе
сборки политики.
	\end{itemize}
	
	\subsubsection{Файлы ядра политики}
	
	\begin{itemize}
	  \item policy/constraints в этом файле определены не MLS
	  ограничения.
	  \item policy/flash/ - Эта директория содержит определения классов и
	  разрешений.
	  \item policy/mls и policy/mcs –Два файла, определяющие конфигурацию для не
	  обязательных MLS особенностей  SELinux.
	  \item \begin{verbatim}policy/global_booleans и policy/global_tunables\end{verbatim} 
	  Два файла, содержащие определения
	  Boolean переменных и их значения по умолчанию.
	  Они компилируются и устанавливаются в /etc/selinux/refpolicy/Booleans и позволяют
	  администратору изменять значения по умолчанию. Они используются 
	  для Условных политик. Причиной двух файлов является то, что в
	  global\_booleans содержатся переменные направленные действительно на
	  поддержку условной политики, администратор может включать и выключать их в рабочей системе.
	  Global\_tunables содержат переменные, которые являются конфигурацией сборки.
	  Они изменяются один раз при установке политики и не меняются после. 
	  \item  policy/modules.conf Этот файл содержит
	  информацию о том, какие модули включаются в процесс сборки и в какой форме. Модули могут собираться в монолитную 
	  политику или в базовые загружаемые модули, или вообще не собираться. Файл modules.conf 
	  создается после выполнения команды make conf.
	  \item \begin{verbatim}policy/modules/\end{verbatim} Директория, содержащая
	  все модули политики, разделенные в разные директории по слоям.  
	  \item \begin{verbatim}policy/support\end{verbatim} эта директория содержит 
	  макросы, используемые модулями для помощи при написании политики. Например
	  файл \begin{verbatim}policy/support/obj_perm_sets.spt\end{verbatim}
	  определяет макрос, который определят список разрешений.
	  \item \begin{verbatim}policy/users\end{verbatim} этот файл содержит все
	  определения пользователей для политики. Он содержит только основные определения. Обычно
	  это system\_u, user\_u и иногда root.
	\end{itemize}
	
	\subsection{Принципы разработки}
	Reference политика сформирована по нескольким принципам проектирования. Эти
	принципы сфокусированы на достижении основной цели проекта. Сейчас большинство
	 из этих принципов выполняются толь по соглашению. 
	 \subsubsection{Слои}
	Одним из основных принципов проектирования политики является строгое разбиение
	ее на модули. Слабым, но не маловажным принципом reference политики является
	разбиение этих модулей на слои. Слои обеспечивают свободную организованную
	структуру для модулей, которые показывают общую архитектуру системы.
	В общем,  в reference политике сделана попытка сохранить зависимость между
	модулями в пределах слоя и более низких слоев. Слои разнесены по каталогам в
	директории policy/modules/. В настоящее время в reference политике определены
	следующие слои:
	\begin{itemize}
	\item Kernel – Ядро. Этот слой содержит модули политики, которые
	непосредственно относятся к ядру Linux. Это самый низкий уровень модулей. Модули
	 в этой слое включают программные инструкции для ядра, устройств, файловой системы,
	  сети. Большинство из этих модулей всегда включается в любой тип политики.
	\item System – это модули, которые так же обычно включаются во все политики, но
	не обращаются к ядру напрямую. Модули в этом слое включаются основные
	библиотеки, процесс аутентификации и управление сетью.
	\item Services – этот слой содержит модули политики для всех сервисов и
	демонов.
	В нем содержатся модули от cron, sshd  до apache.
	\item Admin – этот слой содержит модули политики для утилит администрирования и
	команд, у которых есть свой тип домена.
	\item Apps – этот слой содержит модули политики для всех остальных программ, у
	которых есть свой собственный тип домена и модуль политики.
	Заметим, что разделение на слои не строгое, оно нужно в первую очередь для
	организации наборов модулей.
	\end{itemize} 
	
	\subsubsection{Модульность}
	
	Одним из строгих принципов проектирования reference политики является
	модульность. В reference политике модули должны быть слабо связаны. Эта слабая
	связь осуществляется через исполнение двух принципов проектирования:
	инкапсуляции и абстракции.
	
	\subsubsection{Инкапсуляция}
	
	Инкапсуляция является одним из принципов для достижения модульности. Она
	требует, что бы все типы и атрибуты использовались только в одном модуле. В
	результате, типы и атрибуты не могут быть использованы как глобальные. Только
	модули, которые определяют тип или атрибут могут на него ссылаться напрямую.
	Любой другой модуль должен запросить использование типа или атрибута через
	специальный интерфейс.
	
	Например, в exemple политике, все типы, которые являются типами домена,
	получают атрибут domain. Поэтому в каждом модуле необходимо просто добавлять
	атрибут domain ко всем типам, которые являются типами домена. Если придется
	изменить концепцию, то придется изменять все модули.
	
	В reference модуле, модуль domain в слое ядра определяет концепт домена. Эта
	концепция реализуется так же, как и в exemple политике, однако, если
	потребуется ее изменить, то изменять необходимо будет только в одном месте.
	Любой другой модуль, если хочет объявить один из своих типов доменом, вызывает
	интерфейс, определенный в модуле domain:
	
	domain\_type(my\_type) 
	
	Этот интерфейс объявлен в policy/modules/kernel/domain.if. 
	
	Инкапсуляция позволяет делать реализацию модуля скрытой от остальных модулей с
	помощью интерфейсов.
	\subsubsection{Абстракция}
	
	Абстракция это цель разработки, когда интерфейс раскрывает то, что он делает,
	но не раскрывает как он это делает. Целью интерфейса является описание того,
	какой именно абстракций доступ предоставляется, или какая системная возможность
	включается и этим интерфейсом. Рассмотрим пример ранее показанного интерфейса
	domain\_type(). Целью данного интерфейса является именно превращение типа в тип
	домена. Добавление к типу атрибута domain является скрытой реализацией данного
	интерфейса, поэтому этот интерфейс не называется add\_domain\_attribute(). 
	
	\subsubsection{Файлы модулей}
	
	Как было сказано, когда обсуждалась файловая структура reference политики, все
	модули содержатся в \begin{verbatim}policy/modules/[layer]/\end{verbatim} где
	layer является директорией с именем соответствующего слоя. Каждый модуль должен
	содержать три связанных файла, которые имеют одинаковое имя (без расширения),
	которое является именем модуля:
 
\begin{enumerate}
  \item Внутренний файл политики (.te) Этот файл содержит внутренние объявления
  и правила модуля. Все объявления типов и атрибутов находятся в этом файле и
  правила, которые дают этим типам и атрибутам их права находятся здесь.
  \item Внешний интерфейс (.if) Этот файл содержит интерфейс модуля. Этот
  интерфейс означает, какой доступ получат другие модули к типам и атрибутам
  данного модуля.
  \item Маркировочный файл (.fc) Этот файл содержит инструкции маркирования
  файлов, относящиеся к данному модулю.
\end{enumerate}

Только .te и .if файлы данного модуля могут использовать типы и атрибуты везде.
Все другие модули должны использовать внутренние типы и атрибуты через
интерфейсы модуля.

\subsubsection{Интерфейсы}
	Как уже говорилось ранее, одним из наиболее значимых усовершенствований, 
	добавленных в reference политику, является использование интерфейсов для
	получения доступа к типу из внешних модулей. Интерфейс предоставляет доступ к
	ресурсам модуля (например, приватных типов или атрибутов). Все остальные
	модулю используют этот интерфейс для доступа к этим приватным типам или
	атрибутам. Таким образом, внутренние изменения модуля не вовлекают изменения
	других модулей, как это было в огиринальной политике.
	
	Как отмечалось выше, интерфейсы являются частью модуля и находятся в .if
	файлах. Интерфейсы реализованы в виде макросов. В настоящее время reference
	политика поддерживает два вида интерфейсов: интерфейсы доступа и шаблонные
	интерфейсы.

\section{Синтаксис языка политики SELinux}
    \subsection{Добавление новых классов и разрешений}
    Классы и связанные с ними разрешения является основой контроля доступа в SELinux. Классы представляют собой категории ресурсов, такие как файлы или сокеты, а разрешения представляют собой виды доступа к этим ресурсам, такие как чтение ли отправка. Понимание классов и разрешений является трудным аспектом SELinux, так как требует знание как по SELinux так и по Linux.

	Объекты классов представляют все ресурсы основных типов (например, файлы или
	сокеты). Экземпляры классов (например, конкретный файл или сокет) обычно
	называются просто объект. Иногда термины класс и объект класса используются
	взаимозаменяемо, но важно понимать различие мужду ними. Класса относятся к
	целой категории ресурсов, в то время как объекты указывают на конкретный
	экземпляр класса.
	
	Рассмотрим правило allow:
	\begin{verbatim}
	allow user_t bin_t : file {read execute getattr};
	\end{verbatim}
	В этом правиле процессу с типом user\_t (который является источником или
	предметом) разрешается читать, выполнять и получать атрибуты от всех объектов
	класса файл, у которых указан тип bin\_t и их защищенном контексте. Класса file
	указывается на категорию ресурсов, а bin\_t указывает на экземпляр этой
	категории ресурсов, для которого это правило применимо (это те файлы, тип
	которых file\_t). Это правило не применяется к тем объектам, у которых тип
	bin\_t но класс отличен от file. Так же оно не применятся к файлам, у которых
	тип отличен он bin\_t.
	
	Разрешения в данном правиле read, execute и getattr определяют доступ, который
	разрешается к объекту от предмета, который имеет тип user\_t. Каждое из этих
	разрешений должно быть действительно для указанного класса, и оно должно
	представлять некоторую форму доступа к этому объекту. (Например, разрешение
	read требует использования системных вызовов open(2) и read(2).) Набор
	разрешений определенных для экземпляра класса (так же называемый вектором
	доступа) предоставляет все возможные разрешения, которые позволено выполнять
	над данным объектом.
	
	Набор объектов классов зависят от версии SELinux и ядра Linux. Со временем
	добавляются новые классы, для того что бы обеспечить наиболее полное покрытие
	функционала ядра. Например, новые версии ядра Linux содержат новый вид сокета
	netlink, для логирования фрэймворка. Для таких ядер, которые поддерживают	
	netlink socket, в SELinux добавлен соответствующий класс с необходимыми
	разрешениями.
	
	Политика должна включать объявление всех классов и их разрешений,
	поддерживаемых ядром или другими мэнеджерами объектов. В действительности, для
	того что бы писать политики для конкретных программ нет необходимости
	добавлять новые классы, однако необходимо понимать структуру объявления
	классов, для эффективного написания модулей политики. Понимание синтаксиса
	объявления классов и разрешений, позволяет узнать какие классы и разрешения
	поддерживает данная версия политики.
    
    \subsubsection{Добавление новых классов и разрешений}
    Добавление новых классов и изменение их разрешений задача, требующая
    изменения системного кода. В отличие от других аспектов языка политик
    SELinux, классы и разрешения используются для отображения деталей Linux, в
    частности ядра. На деле получается, что классы и разрешения необходимы для
  того, что бы наиболее правильно передать структуру ресурсов системы. Поэтому
    соответствующие изменения в классах и разрешениях должны следовать за
    изменениями в системе.

 Примером такого изменения может служить добавление новой формы межпроцессорного
 взаимодействия в ядро. В данном случае,  добавляется новая категория ресурсов с
 новыми системными вызовами. И данные ресурсы требуют соответствующего
 представления в SELinux.

 Простое добавление классов или разрешений, не приведет ни к какому эффекту.
 Требуется так же добавить поддержку данного класса в системный код, реализующий
 проверку доступа к объектам данного класса.
     
    \subsubsection{Объявление классов}
    Классы объявляется с помощью инструкции объявления классов, которая имеет
    следующий синтаксис:
    
class class\_name 

class\_name – идентификатор класса. Идентификатор может быть
любой длинны и содержать ASCII буквы и цифры.

Объявление класса описывает только класс и ничего больше. Например, следующая
инструкция описывает объект класса для директории:

class dir 

Инструкция содержит ключевое слово class и следующее за ним имя
класса. Обратите внимание, что инструкция не заканчивается точкой с запятой, как
многие другие инструкции. Имена классов объявляются в отдельном пространстве
имен. Поэтому возможно объявить класс, разрешение и тип с одинаковым именем,
однако это строго не рекомендуется.

Классы могут объявляться только в монолитных политиках или базовых загружаемых
модулях. Объявления классов в не базовых загружаемых модулях не действительны.
Классы нельзя объявлять в условных операторах.
    
    \subsubsection{Объявление и связывание разрешений с классами}
    
    Существует два метода для объявления разрешений. Первый называется «общие
    разрешения» и позволяет нам создавать разрешения, которые в последствии
    можно связать с несколькими классами. Общие разрешения используются для
    похожих классов (например, файлы и символические ссылки). Второй метод
    называется классовые разрешения. Он позволяет указать уникальные разрешения
    для конкретного класса. Как вы увидите, некоторые классы содержат только
    классовые разрешения, а некоторые только общие, и некоторые содержат и те и
    те.
    
    \subsubsection{Общие разрешения}
    Инструкция, объявляющая общие разрешения позволяет создавать список
    разрешений, который можно связать как группу с двумя или более классами. Она
    имеет следующий синтаксис:
\begin{verbatim}
common common_name { perm_set }
\end{verbatim}
common\_name  - идентификатор общих разрешений. Идентификатор может быть любой
длинны и содержать ASCII символы, буквы, тире или точку.

perm\_set – один или несколько идентификаторов разрешений, разделенных пробелами
или переводом строки. Идентификатор может быть любой длинны, содержать ASCII
буквы, цифры, тире или точку.

Объявление общих разрешений возможно только в монолитной политике и базовых
загружаемых модулях. Объявления общих разрешений в условных выражениях или не
базовых загружаемых модулях считается не действительным.

Из UNIX философии «все есть файл» вытекает то, что многие объекты доступа имеют
одинаковые наборы разрешений. В данном случае удобно применять общие разрешения.
Следующий пример объявляет общие разрешения, связанные с файлами:
\begin{verbatim}
common file
{

      ioctl
      read
      write
      create
      getattr
      setattr
      lock

      relabelfrom
      relabelto
      append
      unlink
      link
      rename
      execute
      swapon
      quotaon
      mounton
}
\end{verbatim}
Данная инструкция объявляет общие разрешения, называнные file. Только объявление
общих разрешений не имеет смысла. Их необходимо связать с классом которых их
использует.
Как и с классами, общие разрешения объявляются в своем собственном пространстве
имен. Это может привести к некоторым проблемам, если не быть осторожными.
Например, как показано в предыдущем примере, у нас есть класс и общее
разрешение, названные file. Несмотря на то, что имя одинаковое, на деле это два
разных и сильно отличающихся компонента политики.
    
    \subsubsection{Связывание разрешений и классов}
    Связывание разрешений и классов происходит с помощью инструкции вектора
    доступа. Она имеет следующий синтаксис:
	
	\begin{verbatim}
	class class_name [ inherits common ] [{ perm_set } ]
	\end{verbatim}
	
	class\_name – объявленный ранее идентификатор класса
	
	common – объявленный ранее идентификатор общих разрешений
	
	perm\_set – одно или несколько классовых разрешений. Идентификаторы классовых
разрешений могут быть любой длины и содержать ASCII буквы, цифры, тире или точку.

Как минимум одно разрешение должно быть указано в инструкции, но могут быть
указаны и общие разрешения и классовые вместе в одной инструкции. Результатом
будет пересечение множества всех разрешений.

Объявление инструкций вектора доступа действительно в монолитных политиках и
базовых загружаемых модулях. Их нельзя объявлять в условных выражениях и не
базовых загружаемых модулях.

Следующие пример показывает инструкцию связи одного единственного классового
разрешения с классом dir:
	\begin{verbatim}
	class dir { search }
	\end{verbatim}
Как показывает данный пример, инструкция вектора доступа выглядит похоже на
инструкцию объявления классов (то же ключевое слово class). Объявление классов и
инструкция вектора доступа являются различными инструкциями, начинающимися с
одного ключевого слова. Для инструкции вектора доступа необходимо, что бы класс
был объявлен заранее. Обратите внимание, что инструкция не заканчивается точкой
с запятой.

В предыдущем примере было объявлено единственное разрешение, связанное с
классов. В действительности их может быть несколько:
	\begin{verbatim}
	class dir { search add_name remove_name }
	\end{verbatim}
В данном примере связывается три классовых разрешения с классом dir. Мы может
так же связать общие разрешения, используя ключевое слово inherits в инструкции
вектора доступа. Например, класс dir является одним из так называемых «файловых»
классов. И он включает общие разрешения, свойственные всем файловым классам.
Следующая инструкция вектора доступа является полной инструкцией для класса dir,
она связывает с классом объявленные ранее общие разрешения file и несколько
классово уникальных разрешения для директорий:
	\begin{verbatim}
	class dir
	inherits file
	{

	      add_name
  		  remove_name
	      reparent
	      search
	      rmdir
	}
	\end{verbatim}
Как показано в этом примере, мы использовали ключевое слово inherits, следующее
за именем класса, объявленного ранее, далее идет идентификатор общих разрешений
(file). В результате с классом dir будут связаны все разрешения из общих
разрешений file и 5 классово уникальных разрешений.

Возможно так же использовать только общие разрешения, как в следующем примере:
\begin{verbatim}
class lnk_file inherits file
\end{verbatim}
Данный пример связывает с классом lnk\_file только общие разрешения файлов. 
    
\subsection{Правила вектора контроля доступа}
    
    Вектор доступа определяют свои значения в зависимости от разрешений для
    объекта класса. Язык политики SELinux сейчас поддерживает четыре типа правил
    вектора доступа:

allow – определяет доступ разрешения между двумя типами

dontaudit – определяет отказ в доступе

auditallow – определяет разрешение на запись события

neverallow – определяет те права доступа, которые не могут  быть предоставлены в
любом из модулей.

 Мы рассмотрим каждое из этих правил, их общий и уникальный синтаксис и
 семантику и примеры их использования.

\subsubsection{Общий синтаксис правил вектора доступа}

Несмотря на то что каждое правило вектора доступа имеет различные цели, у них у
всех общий синтаксис. Каждое правило состоит из пяти элементов:

\begin{enumerate}
  \item Имя правила – allow, dontaudit, auditallow, neverallow
  \item Тип источника – тип которому предоставляется доступ, обычно это тип
  домена процесса который пытается получить доступ.
  \item Целевой тип – тип(ы) объекта, на который источник получает доступ.
  \item Класс объекта(ов) – класс(ы) объекта(ов) , которому предоставляется
  доступ.
  \item Разрешение – конкретные права доступа, указывающие что именно
  разрешается источнику.
\end{enumerate}
    
Простое правило вектора доступа содержит один тип источника, один целевой тип,
класс объекта и разрешение. Пример такого правила:

\begin{verbatim}
allow user_t bin_t : file execute;
\end{verbatim}

Это правило allow содержит тип источника user\_t, целевой тип bin\_t, объект
класса file, и разрешение execute. Это правило следует читать как «Разрешить
user\_t запускать файлы типа bin\_t”.

Все четыре правила вектора доступа обычно имеют похожий синтаксис с разными
именами ключевых слов. Например, мы можем преобразовать предыдущий пример в
auditallow правило, просто заменив ключевое слово:

\begin{verbatim}
auditallow user_t bin_t : file execute;
\end{verbatim}

О том, что значит данное правило будет сказано ниже. Что важно в данный момент,
так  это то, что синтаксис действительно одинаковый.
    
\subsubsection{Ключи вектора доступа}

Внутри ядра все правила вектора доступа уникально идентифицируются триплетов из
типа источника, целевого типа и объекта класса. Этот триплет называется ключом,
используемым для хэш таблицы и содержат ключ внутри данных политики. Правила
ищутся и хранятся по данному ключу. Когда процесс совершает запрос на доступ,
модуль LSM SELinux запрашивает доступ основываясь на этом ключе.

Итак, что случится когда появляется несколько векторов доступа с одинаковым
ключом (т.е. содержат тот же тип источника, целевой тип и объект класса).
Например, рассмотрим политику со следующими правилами:
    
\begin{verbatim}
allow user_t bin_t : file execute;
allow user_t bin_t : file read;
\end{verbatim}    

Процессу типа user\_t позволяется чтение или выполнение файлов типа bin\_t?
Позволяется и то и то. Все правила с одинаковым ключом комбинируются при
выполнении checkpolicy. Скомпилированная политика содержит единственное правило
с обоими и execute и read разрешениями. Все правила вектора доступа складываются
таким образом.

Каждое последующее правило вектора доступа в политике, которое имеет те же ключи
что и предыдущие, добавляет разрешения на конечное правило в собранную политику.

Не существует понятия удаления разрешения, предоставленного другим правилом. Так
что будьте осторожны, вы можете написать одно разрешения в одной части политики,
другое в другой, а в конечной политике будут содержаться оба разрешения.

\subsubsection{Использование атрибутов в правилах вектора доступа}

Несмотря на то, что правила вектора доступа, которые мы видели до сих пор были
простые, синтаксис поддерживает множество способов перечисления типов, объектов
класса и разрешений. Это позволяет нам быть более гибкими при создании политик и
делать инструкции более короткими.

В простой форме правила в предыдущем примере, правило упоминает непосредственно
тип источника и целевой тип. Но часто бывает удобно, однако,  указать несколько
исходных или целевых типов. Одним из способов для обозначения различных типов
является использование атрибутов. Атрибуты можно использовать везде, где можно
использовать тип в правиле вектора доступа.

Например, предположим, что мы определили атрибут exec\_type, который мы
планируем связать со всеми типами программ обычного пользователя (тип домена user\_t) для
того, что бы разрешить ему запуск. Теперь мы можем поменять предыдущий пример и
сослаться на атрибут exec\_type вместо явного типа bin\_t, как показано здесь:

\begin{verbatim}
allow user_t exec_type : file execute;
\end{verbatim}

В отличие от предыдущего примера, это правило не прямо отражает то, что будет
отражаться в ядре. Правила, которые включают атрибуты будут развернуты внутри
ядра в отдельное правило для каждого типа, связанного с атрибутом. Если типов
было 20, то в конечной политике появится 20 ключей, предоставляющих доступ
каждому из типов.

Мы так же можем использовать атрибуты в качестве типа источника, или вместо
обоих типов. Например, предположим, мы также создали атрибут (domain) который мы
связали со всеми нужными  типами доменов (включая user\_t), и которым мы
предоставляем доступ на выполнение файлов с атрибутом file\_type, Мы можем
достичь этой цели одним правилом:

\begin{verbatim}
allow domain exec_type : file execute;
\end{verbatim}

Для того что бы лучше проиллюстрировать концепцию расширения правил, представим,
что наша политика, связывает атрибут типов user\_t и  тип staff\_t с типами 
exec\_type, bin\_t,  local\_bin\_t  и sbint\_t. Таким образом, одно правило
эквивалентно следующими правилам:

\begin{verbatim}
allow user_t bin_t : file execute;
allow user_t local_bin_t : file execute;
allow user_t sbin_t : file execute;
allow staff_t bin_t : file execute;
allow staff_t local_bin_t : file execute;
allow staff_t sbin_t : file execute;
\end{verbatim}

\subsubsection{Множественные типы и атрибуты в правиле вектора доступа}

Мы не ограничиваем одним типом или атрибутом тип источника или целевой тип. Мы можем указывать список типов или атрибутов как для типа источника так и для целевого типа. Список разделяется пробелами и заключен в фигурные скобки, как показано в следующем примере:

\begin{verbatim}
allow user_t { bin_t sbin_t } : file execute;
\end{verbatim}

В этом правиле, целевым типом являются являются оба и bin\_t и sbint\_t типа.
Правило с несколькими типами или атрибутами преобразуются в несколько правил на
этапе компиляции. В предыдущем примере, скомплированная политика будется
содержать два ключа, для каждого из целевых типов.

Мы можем использовать списки типов или атрибутов как для целевого типа так и для
типа источника, или для обоих сразу. Например, следующее правило совершенно
законно:

\begin{verbatim}
allow {user_t domain} {bin_t file_type sbin_t} : file execute ;
\end{verbatim}

Политика ядра будет содержать по одному ключу для каждой комбинации исходного
типа и целевого типа.

\subsubsection{Специальный тип self}

Язык политики содержит ключевое слово self, которое используется в качестве
целевого типа. Например, следующие два правила эквивалентны:

\begin{verbatim}
allow user_t user_t : process signal;
allow user_t self : process signal;
\end{verbatim}

Ключевое слово self указывает политике предоставить доступ для каждого исходного
типа самому себе. В предыдущем примере второе правило всего лишь создает ключ, в
котором целевой и исходный типы user\_t.

Рассмотрим более сложный пример:

\begin{verbatim}
allow {user_t staff_t} self : process signal;
\end{verbatim}

В этом примере правило создает два ключа, по одному на каждый из исходных типов.
Это правило эквивалентно двум следующим:

\begin{verbatim}
allow user_t user_t : process signal;
allow staff_t staff_t : process signal;
\end{verbatim}

Обратите внимание, что когда используете ключевое слово self, эквивалентные
правила создаются только для каждого исходного типа. Т.е. user\_t не получает
доступ к staff\_t и наоборот.

Так же стоит обратить внимание, что ключевое слово self может использоваться
только на месте целевого поля в правиле вектора доступа. В частности, вы не
можете использовать self как исходный тип.  Кроме того вы не можете объявить тип
или атрибут с идентификатором self.

Использование self особенно ценно, при использовании атрибутов или больших
списков типов в качестве источника правила. Например, предположим, мы хотим, что
бы каждый домен имел возможность посылать самому себе сигнал. Мы могли бы
написать правило подобное этому:

\begin{verbatim}
allow domain domain : process signal;
\end{verbatim}

Но это не то, что мы хотим в действительности. В данном примере кроме того, что
каждый домен может посылать сигнал себе, он так же может посылать сигнал и
каждому процессу из домена domain. Используя ключевое слово self можно исправить
данную проблему:

\begin{verbatim}
allow domain self : process signal;
\end{verbatim}

\subsubsection{Специальный оператор отрицания}

Последним элементом синтаксиса для типов в правиле вектора доступа является
оператор отрицания. Этот синтаксис позволяет удалить из списка тип и наиболее
часто используется для удаления типа из атрибута в данном правиле. Это
осуществляется, путем указывания перед типом знака минус: -. Например, мы можем
позволить всем типам домена выполнять файл атрибута exec\_type за исключением
sbin\_t в следующем правиле:

\begin{verbatim}
allow domain { exec_type -sbin_t } : file execute;
\end{verbatim}

Это правило будет выполняться, даже если атрибут exec\_type не содержит тип
sbin\_t.
Порядок указания типа и атрибута неважен. Следующее правило эквивалентно предыдущему:

\begin{verbatim}
allow domain { -sbin_t exec_type } : file execute;
\end{verbatim}

\subsubsection{Указание класса объекта и разрешений в правиле вектора доступа}

Правило вектора доступа может так же содержать список классов и разрешений.
Синтаксис аналогичен описанию типов, список разделенный пробелами и заключенный
в фигурные скобки.

\begin{verbatim}
allow user_t bin_t : { file dir } { read getattr };
\end{verbatim}

Это правило в результате преобразуется в два ключа, по одному на каждый класс,
так же как с типами источника или целевыми типами.

\begin{verbatim}
allow user_t bin_t : file { read getattr };
allow user_t bin_t : dir { read getattr };
\end{verbatim}

Обратите внимание, что список классов разширяется, но каждое правило содержит
обинаковый список разрешений. Это значит, что все указанные разрешений должны
быть действительны для всех классов, указанных в правиле. Иногда приходится
создавать два одиночных правила с одинаковыми исходными и целевыми типами,
потому что классы указанные в правиле, имеют несовместимый набор разрешений.
Например, если мы посмотрим на разрешений классов file и dir мы заметим, что
большинство из них одинаковые, но некоторые нет. (Разрешения действительные для
обоих являются результатом использования общих разрешений).

Предположим, например, мы хотим написать правило предоставляющее доступ на
чтение обоим классам. Следующее правило не является правильным:

\begin{verbatim}
allow user_t bin_t : { file dir } { read getattr search };
\end{verbatim}

Несмотря на то, что read и getattr являются общими разрешениями для обоих
классов, разрешение search есть только у класса dir. Из-за этого checkpolicy не
может создать ключ, который предоставит классу file разрешение search, и мы
получим ошибку, когда попытаемся скомпилировать это правило. Нам придется
создать два правила, такие как следующие:

\begin{verbatim}
allow user_t bin_t : file { read getattr };
allow user_t bin_t : dir { read getattr search } ;
\end{verbatim}

\subsubsection{Специальные операторы разрешений}

Мы можем использовать два специальных оператора для указания разрешений в
правиле вектора доступа. Первый оператор звездочка *. Означает все разрешения,
доступные данному классу:

\begin{verbatim}
allow user_t bin_t : { file dir } *;
\end{verbatim}

Это правило разворачивается в все разрешения для классов file и dir.

Как видно из примера, данный оператор отличается от простого перечисления тем,
что может указывать разные разрешения для разных классов. Это позволяет
использовать оператор  * в правилах с несколькими классами, даже если классы
имеют различные разрешения. Таким образом, правило выше предоставляет все
разрешения для класса dir и все разрешения для класса file.

Второй специальный оператор позволяет указать все разрешения на исключением
какого-либо списка, благодаря оператор тильда (~):

\begin{verbatim}
allow user_t bin_t : file ~{ write setattr ioctl };
\end{verbatim}

При компиляции данное правило предоставлет все разрешения класса file за
исключением write, setattr, ioctl. Так же как и оператор *, данный оператор
предоставляет доступ индивидуально для каждого из указанных классов.

Имейте в виду, что разрешено использовать данные специальные операторы и в
других случаях. Данная возможность появилась в последних версиях компилятора.
Многие последние версии checkpolicy позволяют, например, использовать оператор *
для типов.

\subsubsection{Общий синтаксис правил вектора доступа}

Полный общий синтаксис следующий:

\begin{verbatim}
rule_name  type_set  type_set : class_set  perm_set ;
\end{verbatim}
\begin{description}
\item rule\_name – Имя правила. Позволяется использовать allow, auditallow,
auditdeny, dontaudit, neverallow ключевые слова.

\item type\_set – Один или несколько типов или атрибутов. Задаются
отдельные списки для исходных типов и целевых типов. Несколько типов и атрибутов
разделяются пробелами и заключаются в фигурные скобки, например 
\begin{verbatim}{bin_t, sbin_t}\end{verbatim}
Типы могут быть исключены из списка, путем использования оператора -. Например,

\begin{verbatim}{exec_type –sbin_t}\end{verbatim}. Ключевое слово self может
использовать в качестве целевого типа и не может в качестве исходного. Правило neverallow так же позволяет
использовать оператор * для включения всех типов и оператор ~ для включения всех
типов за исключением списка.

\item class\_set – Один или несколько классов. Несколько классов разделяются
пробелами и заключаются в фигурные скобки, например 
\begin{verbatim}{file, lnk_file}\end{verbatim}

\item perms\_set – Одно или несколько разрешений. Все разрешения должны быть
действительными для всех объектов класса, указанных в class\_set. Несколько
разрешений должны быть заключены в фигруные скобки, например 
\begin{verbatim}{read create}\end{verbatim}
Оператор * означает список из всех разрешений, доступных данному классу.
Оператор ~ означает список из всех разрешений, за исключением указанных.

\end{description}

Все правила вектора доступа можно указывать в монолитной политике, базовых
загружаемых модулях и не базовых загружаемых модулях. Все правила, за
исключением auditdeny и neverallow могут быть использованы в условных
операторах.

\subsubsection{Правило allow}

Вы могли уже не раз видеть примеры данного правила в этой работе. Правило allow
наиболее популярное правило в политиках.

Как уже рассказывалось, правило allow указывает все разрешения, которые доступны
домену. Запомните, никакое разрешение не дается по умолчанию. Мы явно указать
разрешения между двумя списками типов – исходных и целевых, для списка заданных
классов, например:

\begin{verbatim}
allow user_t bin_t : file { read execute };
\end{verbatim}

Данное правило позволяет любому процессу, чей контекст безопастности содержит
тип user\_t, разрешить чтение и выполнение любого обычного файла, у которого тип
bin\_t. Правило allow поддерживает весь общий синтаксис правил вектора доступа и
не имеет ни какого дополнительного синтаксиса.

Как и все правила вектора доступа, allow правило в загруженной политике является
пересечением всех правил с одинаковым ключом (объект, цель, класс). Например,
следующие две инструкции:

\begin{verbatim}
allow user_t bin_t : file read;
allow user_t bin_t : file write;
\end{verbatim}

Эквиваленты одной:

\begin{verbatim}
allow user_t bin_t : file { read write );
\end{verbatim}

\subsubsection{Правило Audit}

SELinux содержит широкие возможности для регистрации и проверки попыток доступа,
которые разрешены или запрещены политикой. Сообщения проверки, так же называемые
«AVC сообщения» дают детальную информацию о попытке доступа, разрешен был доступ
или нет, контекст домена и контекст цели, и другую детальную информацию о
ресурсах, вовлеченных в попытку доступа. Сообщения, которые похожи на другие
сообщения в ядре, обычно сохраняются в лог файл в директории /var/log. Они
являются незаменимым инструментом для разработки, системного администрирования и
мониторинга. Правило audit указывает для каких попыток доступа следует
генерировать сообщения.

По умолчанию SELinux не записывает никакие попытки доступа, для которых доступ
был разрешен, однако записывает все разрешенные попытки доступа. Данное
поведение оправдано, так как на многих система обрабатываются тысячи обращений
проверки доступа в секунду, и только нескольким из них доступ запрещается. Язык
политик позволяет частично переписать данное поведение по умолчанию. Благодаря
этому можно генерировать сообщения о некоторых разрешенных попытках доступа, и,
наоборот, не генерировать сообщения об определенных запретах доступа.

В SELinux существует два правила, которые позволяют нам контролировать
логирование попыток доступа: dontaudit и auditallow. Эти два правила позволяют
нам изменять настройки по умолчанию логирования попыток доступа. Наиболее часто
используется правило dontaudit. Оно указывает те события, которые не стоит
записывать в лог, несмотря на то, что доступ был запрещен.

Сообщения о запрете доступа генерируются только тогда, когда доступ запрещен
модулем SELinux, который является LSM модулем. Проверка в LSM модуле
осуществляется только тогда, когда стандартная проверка доступа Linux доступ
разрешила. Это означает, что если доступ заблокирован стандартной проверкой
доступа Linux, то сообщение о запрете доступа не будет сгенерировано. Однако,
если необходимо отлавливать и подобные случаи, то можно использовать напрямую
систему логирования ядра, включенную в 2.6.х версии ядра. Для большей информации
смотрите мануал по auditd(8) и auditctl(8).

Рассмотрим пример отмены логирования сообщения о запрете доступа:

\begin{verbatim}
dontaudit httpd_t etc_t : dir search;
\end{verbatim}

Данное правило указывает, что когда процессу типа httpd\_t запрещено производить
поиск в директории типа etc\_t, сообщение о запрете не будет сохраняться,
несмотря на то, что по умолчанию данное сообщение должно быть сохранено. Правило
dontaudit обычно используется в тех случаях, когда запрет доступа приложению
является ожидаемым. Оно позволяет избежать большого количества сообщения в
системных журналах.

Другое правило логирование, auditallow, позволяет нам записывать разрешенные
попытки доступа, которые не логируются по умолчанию. Например, давайте посмотрим
на следующее правило:

\begin{verbatim}
auditallow domain shadow_t : file write;
\end{verbatim}

Данное правило показывает, что когда процесс с типом domain получает доступ на
запись в файл типа shadow\_t , сообщение о данном сообщении будет записано в
лог. Правило auditallow используется для логирования важных событий, например
доступа на запись к файлам с паролями, или загрузку новой политики в ядро.

Запомните, что правила логирования перезаписывают поведение по умолчанию.
Правило allow указывает, какой доступ разрешен, а какой нет. Auditallow не
выполняет данных действий, оно только включает запись событий.

Обратите внимание, что логирование отличается в permissive и enforcing режимах.
Когда запущен enforcing режим, сообщения о запрете доступа генерируются каждый
раз, когда возникает событие, требующее логирования. В permissive режиме
сообщения генерируются только один раз для каждого события, до следующей
загрузки политики или до включения режима enforcing. Permissive режим наиболее
часто используется для разработки модулей политики и данное поведение позволяет
уменьшить размер логов.

\subsubsection{Правило neverallow}

Последним правилом вектора доступа является neverallow. Данное правило
используется для указания тех правил доступа, которые никогда нельзя будет
указать с помощью правила allow. По умолчанию доступ запрещен для всех тех
случаев, которые не разрешены правилом allow. Правило neverallow существует не
для запрета доступа, а для запреда создания соответствующих правил allow, для
того, что бы явно заблокировать нежелательные разрешения. Напомним, что политика
может содержать десятки тысяч правил доступа. Поэтому можно случайно указать тот
доступ, который указывать не следуют. Правило neverallow помогает предотвратить
подобные случаи.

Рассмотрим пример правила:

\begin{verbatim}
neverallow user_t shadow_t : file write;
\end{verbatim}

Данное правило предотвращает добавления правила доступа, которое позволяет
домену user\_t писать в файлы типа shadow\_t, генерируя ошибку компиляции.
Данное правило не удаляет доступ, оно только генерирует ошибку компиляции.

Правило neverallow поддерживает некоторый дополнительный синтаксис, который
расширяет общий синтаксис правил вектора доступа. В частности можно использовать
операторы * и ~ в поле указания целевого и исходного типов. Данные операторы
работают так же как и со списком разрешений в других правилах вектора доступа.
Например, следующее правило:

\begin{verbatim}
neverallow * domain : dir ~{ read getattr };
\end{verbatim}

Данное правило указывает, что не может быть ни каких правил allow которые
разрешают любому типу домена любой доступ к директории типа, включенного в
атрибут domain, за исключением доступа на запись и получения атрибутов.
Звездочка в данном случае означает все типы.

Правило neverallow, похожее на это используется в политике для предотвращения
нежелательного доступа к директориям /proc/ которые содержат информацию о
процессах.

Другой вариант правила следующий:

\begin{verbatim}
neverallow domain ~domain : process transition;
\end{verbatim}

Данное правило указывает, что процесс типа с атрибутом domain, не может
переходить в тип, который не имеет атрибута domain.

\subsection{Правила типов}

Правила типов определяют  правила создания или изменения типов объектов.
Существует два правила, определенных в языке политик:

\begin{description}
\item type\_transition – определяет поведение по умолчанию для перехода процесса
из домена в домен и создания объектов.
\item type\_change – задает правила изменения типа объектам из SELinux
приложений.
\end{description}

В отличие от правил вектора доступа, данные правила вместо списка разрешений
определяются тип.

\subsubsection{Общий синтаксис правил типов.}

Как и правила вектора доступа, правила типов имеют некоторый общий синтаксис. Каждое правило содержит пять элементов:
\begin{enumerate}
\item Тип правила – type\_transition или type\_change
\item Исходный тип – исходный тип процесса
\item Целевой тип – тип объекта, для которого создается новый тип или меняется
старый Класс объекта – класс объекта, тип которого создается или меняется
\item Тип по умолчанию – единственный тип, для нового объекта или измененного.
\end{enumerate}

Большая часть синтаксиса похожа на синтаксис правил вектора доступа. Но существуют важные различия. Первое – нет списка разрешений. В отличие от правил вектора доступа, правила типа не предоставляют доступ или логировние, поэтому им не нужны разрешения. Вторым важным отличием является то что класс не связан с целевым типов. Класс связан с объектом, который которому будет назначен тип по умолчанию.

Простым примером правила типа является следующий:

\begin{verbatim}
type_transition user\_t passwd\_exec\_t : process passwd\_t;
\end{verbatim}

Данное правило показывает, что когда процесс типа user\_t запускает файлы типа
passwd\_exec\_t, тип процесса пытается измениться, по умолчанию, на passwd\_t.
Целевой тип связан с классом file, тогда как запускаемый класс process. Запускаемый класс (process) связан с исходным типом и типом по умолчанию. Эту ассоциацию легко не заметить, даже если вы уже опытный разработчик политик.

Как и с правилами вектора доступа мы можем указывать несколько классов,
используя разделенный пробелами список, заключенный в фигурные скобки. Так же мы
можем использовать атрибуты и списки типов в полях целевого и исходного типов:

\begin{verbatim}
type_transition { user_t sysadm_t } passwd_exec_t : process passwd_t;
\end{verbatim}

Данное правило включает два типа, user\_t и sysadm\_t, в списке исходных типов.
Как и правила вектора доступа, данное правило будет преобразовано в два правила.
Предыдущий пример имеет тот же смысл что и следующие два правила:

\begin{verbatim}
type_transition user_t passwd_exec_t : process passwd_t;
type_transition sysadm_t passwd_exec_t : process passwd_t;
\end{verbatim}

Использование атрибутов работает так же как в правилах вектора доступа.

В отличие от целевого и исходного типов, в качестве типа по умолчанию не может
использовать атрибут или список типов. Если определить в качестве типа по
умолчанию несколько типов, то ядро не сможет выбрать какой именно тип присвоить
объекту после выполнения данного правила. Из-за этого не может быть более одного
правила с одинаковыми исходным типов, целевым типов и классом. Следующие два
примера вызовут конфликт:

\begin{verbatim}
type_transition user_t passwd_exec_t : process passwd_t;
type_transition user_t passwd_exec_t : process user_passwd_t;
\end{verbatim}

Компилятор политики сгенерирует ошибку для обоих из этих правил, присутствующих
в политике.
\subsubsection{Общий синтаксис:}
\begin{verbatim}
rule_name  type_set  type_set : class_set  default_type;
\end{verbatim}
\begin{description}
\item rule\_name – имя правила типа. 
\item type\_set – один или несколько типов. Это список исходных и список целевых
типов. Несколько типов или атрибутов разделены пробелами и объединены в фигурные
скобки, например { bin\_t sbin\_t }. Типы могут быть исключены из списка с
помощью оператора – перед именем типа (например { exec\_type -sbin\_t }).
\item class\_set – один или несколько классов. Несколько классов разделены
пробелами и объединены в фигурные скобки, например { file lnk\_file }.
\item default\_type – один тип, который задается по умолчанию для создаваемого
объекта или для изменяющегося объекта. Атрибут или список типов не может быть
использован.
\end{description}

Все правила типов действительны в монолитных политиках, базовых загружаемых
модулях и не базовых загружаемых модулях, а так же в условных выражениях.

\subsubsection{Правила перехода типов}

Правило type\_transition используется для того, что бы указать тип по умолчанию
для определенных событий. В настоящее время существует две формы правила
type\_transition. Первая обрабатывает события перехода из домена в домен. Вторая
форма этого правила обрабатывает переход объектов, которая позволяет нам указать
явно тип по умолчанию для новых объектов.

Обе формы помогают быть расширенной безопасности SELinux быть более прозрачной
для обычного пользователя. В SELinux по умолчанию, вновь созданные объекты в
директории наследуют тип объекта, содержащего их (например, директории), а
процесс наследует тип родительского процесса. Правила typte\_transition
позволяет нам изменить данное поведение по умолчанию. Это используется, к примеру, для
того, что бы когда программа работы с паролями создает свои файлы в директории
/tmp, эти файлы имели отличный тип от тех, которые создает обычный пользователь.

Правила type\_transition не предоставляют доступ. Они обеспечивают только новое
правило маркирования объектов по умолчанию. Успешное событие перехода типа
всегда требует связанного с ним набора allow правил, которые позволяют создавать
объекты и маркировать их. Добавим, что поведение по умолчанию, которое
описывается в правиле type\_transition производит какой-либо эффект, только если
созданный процесс не переопределяет данное поведение.

\subsubsection{Правила по умолчанию для перехода доменов}

Рассмотрим подробнее одну из форм правила type\_transition, а именно правила
перехода доменов. Домен изменяет тип, когда запускает файл на выполнение.
Например, взглянем на следующий пример:

\begin{verbatim}
type_transition init_t apache_exec_t : process apache_t;
\end{verbatim}

В данном type\_transition правиле говорится, что когда процессы типа init\_t
выполняют файл типа apache\_exec\_t, тип процесса должен измениться на
apache\_t. Класс в данном правиле указывает, что это правило перехода домена.
Правила перехода домена на самом деле изменяют тип существующего процесса вместо
маркирования вновь созданного. Это связано с тем, что в Linux новый процесс
создается вначале точной копией родительского процесса с помощью вызова fork().

Последние версии SELinux добавляют к классу process разрешение dyntransition.
Это разрешение, которое было добавлено в основном для совместимости с другими
системами, позволяет процессу изменять тип домена еще в запросе, а не только
после выполнения. Данный тип перехода процесса не является безопасным, поскольку
он позволяет вызывающему домену выполнять произвольный код в новом домене,
стирая различия между двумя доменами. Рекомендуется никогда не использовать
данное разрешение в политике, за исключением случаев, когда вы точно уверены что
без этого не обойтись.

Как было замечено ранее, переход типа может произойти только если политика
позволяет соответствующий доступ. Для успешной смены домена политика должна
включать три разрешения:
\begin{description}
\item Execute – исходный тип должен иметь разрешение на выполнение файлов
целевого типа.
\item Transition – исходный домен должен иметь разрешение на переход к типу по
умолчанию 
\item Entrypoint – новый домен (по умолчанию) должен иметь entrypoint разрешение
на файлы с целевым типом apache\_exec\_t.
\end{description}

Таким образом, правило перехода домена, указанное выше, требует следующих allow
правил, для успешного выполнения:
\begin{verbatim}
allow init_t apache_exec_t : file execute;
allow init_t apache_t : process transition;
allow apache_t apache_exec_t : file entrypoint;
\end{verbatim}
На практике обычно данные правила стоит дополнить еще несколькими. Например,
разрешение уведомлять исходный тип из домена типа по умолчанию при его
завершении (это sigchld разрешение), наследование дескрипторов файлов, а также
разрешение на общение процессов через pipe.

\subsubsection{Правила перехода объектов}

Правила переходов объектов указывают тип по умолчанию для вновь созданных
объектов. На практике вы обычно используем данную форму правила time\_transition
в основном для объектов файловой системы (например, file, dir, lnk\_file). Так
же как и переход доменов, эти правила указывают только тип по умолчанию для
перехода. Успешность перехода зависит от того, есть ли в политике необходимые
правила allow.

Пример правила перехода объектов:
\begin{verbatim}
type_transition passwd_t tmp_t : file passwd_tmp_t;
\end{verbatim}
Данное type\_transition правило означает, что когда процесс типа passwd\_t
создает обыкновенный файл (file object class) в директории типа tmp\_t, файлу,
по умолчанию, присваивается тип passwd\_tmp\_t, если это разрешено политикой.
Обратите внимание, что в данном случае класс связан только с типом по умолчанию
(passwd\_tmp\_t). В данном правиле, tmp\_t неявно связаны с объектом класса dir,
потому что это единственный класс, который может содержать файлы. Также, каа и
прежде, политики должны разрешить маркирование файлов, для того что бы задание
типа файлу произошло. Доступ, необходимый для того, что бы маркирование было
выполнено включает разрешения add\_name, write и search для директории типа
tmp\_t, и разрешения write и create для файлов типа passwd\_tmp\_t.

Данный пример показывает технику для решения проблем безопасности в директориях,
которые предназначены для файлов многих приложений, таких как временные
директории. Правила перехода типов используются для любых объектов, которые
создаются при выполнении и которые должны иметь тип отличный от родительского.

При некоторых условиях правила перехода типов не могут быть исполнены. Когда
процессу необходимо создать файл с различными типами в одном контейнере,
type\_transition правила не достаточно. Например, рассмотрим процесс, который
создает два сокета и директории /tmp, которая используется многими другими
доменами для общения. Если мы хотим, что бы файлы сокета имели различный тип,
правила type\_transition будет не достаточно. Мы получим два правила с
одинаковыми типами источника, назначения и классом, но разным  типами по
умолчанию, что приведет к ошибке компиляции. Решением данной проблемы может быть
создание файлов сокета при установке и четкого обозначения типов для них, или
создания сокетов в различных директориях, или непосредственный запрос на задание
типа сокета.

\subsubsection{Правила изменения типа}

Правила type\_change используется для того, что бы указать типы по умолчанию для
переобозначения в SELinux-совместимых приложениях. Как и type\_transition
правила, правила type\_change определяют только маркировку по умолчанию, но не
разрешают доступ. В отличие от type\_transition правил, результат type\_change
не отражается в ядре, он необходим для пользовательских приложений, каких как login
или sshd, которые меняют метки на основе политики. Например, рассмотрим
следующее правило:
\begin{verbatim}
type_change sysadm_t tty_device_t : chr_file sysadm_tty_device_t;
\end{verbatim}
Это type\_change правило показывает, что когда происходит процедура
перемаркирования файлов от имени sysadm\_t файлы класса chr\_file должны
получить тип sysadm\_tty\_device\_t. Данное правило является наиболее распространенным
правилом type\_change в политике. Оно необходимо для перемаркирования терминала
при входе пользователя в систему. При входе в программу она делает запрос
политики через интерфейс модуля ядра SELinux, передавая типы sysamd\_t и
tty\_device\_t и получая sysadm\_tty\_device\_t как результирующий тип. Этот
механизм позволяет при входе в систему для обозначения устройства терминала от имени
пользователя использовать политики заключенные в ядро, а не жестко задавать тип
в программе.

\subsubsection{Правило type\_member}

Компилятор политики так же поддерживает третье правило типов, type\_member. В
настоящее время это правило не имеет смыслового значения, и его использование не
произведет никакого эффекта. Работа по этому правилу еще не завершена. Одна в
будущем данное правило будет предназначено для указания типов объектам
polyinstantiated. Синтаксис правила будет аналогичен двум другим правилам типов.

    
\section{Процесс добавлениея нового класса доступа} 
	\subsection{Настройка окружения и получение исходных кодов политики}
	Реализации SELinux модуля существуют для большинства популярных дистрибутивов
	Linux. Для Red Hat Enterprise Linux SELinux доступен с коммерческой поддержкой,
	для остальных дистрибутивов поддержка производится сообществом. В работе
	использовался дистрибутив Fedora 16, как один из самых популярных дистрибутивов
	Linux. Второй причиной использования именно этого дистрибутива, является то,
	что по умолчанию в дистрибутив уже включен модуль SELinux, а так же targeted
	политика. Так же для данного дистрибутива существует множество утилит для
	администрирования, настройки и написания политик и их модулей. 
	
	В дистрибутиве Fedora 16 по умолчанию присутствует политика SELinux, однако
	исходные коды ее необходимо скачивать отдельно. Как уже говорилось в первой
	части, для работы выбрата версия политики от компании Tresys technology,
	которая называется Reference Policy. Причиной данного выбора стало то, что
	политика используемая в выбранном дистрибутиве основана на reference политике,
	а так же то, что разработка данной политики продолжается и в нее добавляется
	новый функционал и правки. Исходные коды политики можно скачать с git
	репозитория, выполнив последовательность команд:
	\begin{verbatim}
	$ git clone http://oss.tresys.com/git/refpolicy.git
	$ cd refpolicy
	$ git submodule init
	$ git submodule update
	\end{verbatim}
	Несмотря на то, что по умолчанию c Fedora поставляется и политика SELinux, их
	можно поставить и отдельно, либо обновить существующие версии:
	\begin{verbatim}
	yum install selinux-policy selinux-policy-targeted
	\end{verbatim}
	Ряд утилит, использующися при работе с SELinux включены в пакет coreutils.
	Часть команд используется и при работе с обычной версией Linux, но в них
	добавлен функционал для поддержки SELinux. Так, например, команды ls и ps
	позволяют просматривать соотвественно контектсы безопастности файлов в
	директории или процессов.
	
	Для компиляции своих модулей, администрирования и просмотра информации о
	установленных политиках необходимо установить ряд пакетов, которые можно найти
	в репозиториях Fedora:
	\begin{verbatim}
	yum install policycoreutils setools libselinux checkpolicy
	\end{verbatim}
Описание наиболее часто употребляющихся утилит и пакеты, в которых их можно
найти представлены в таблице
	
	\begin{longtable}{|p{3cm}|p{11cm}|p{3cm}|} 
	\caption[Утилиты SELinux]{Утилиты SELinux}\label{tag:tableutilsselinux} \\
	\hline
	\textbf{Название} & \textbf{Описание} & \textbf{Пакет} \\
	\endfirsthead
	\hline
	apol & Предоставляет графический интерфейс для просмотра существующих правил
	политики. & setools
	\\
	\hline
	audit2allow & необходима для чтения лога и отображения правил & policycoreutils
	\\
	\hline
	audit2why & отображает почему доступ запрещен & policycoreutils \\
	\hline
	checkmodule & создает модуль политики из исходных кодов & checkpolicy \\
	\hline
	checkpolicy & создает базовую политики из исходных кодов & checkpolicy \\
	\hline
	getfilecon & позволяет просмотривать контекст файла по пути & libselinux \\
	\hline
	getseuser & отображает информацию о пользователях SELinux & libselinux \\
	\hline
	seinfo & Отображает такую информацию о загруженной политике, как список типов,
	список пользователей, список классов и т.д. & libselinux \\
	\hline
	semanage & Администрирует такие аспекты SELinux, как порты, интерфейсы,
	контексты файлов, логические параметры, роли и уровни для пользователей, MLS
	переходы и существующие типы & policycoreutils \\
	\hline
	semodule & устанавливает, удаляет и отображает список модулей в
	операционной системе & policycoreutils \\
	\hline
	sestatus & отображает инфрмацию о запущенном SELinux & policycoreutils \\
	
	
	\hline	
	\end{longtable}
	
	Для проверки доступа потребуется библиотека libselinux. Она доступна для
	нескольких языков программирования, я использовал ее версию для СИ.
	Устанавливается она вместе с пакетом libselinux.
	
	\subsection{Добавление класса и его разрешений}
	Добавление новых классов возможно только в основную политику. Поэтому для того,
	что бы добавить новый класс доступа приходится перекомпилировать и
	перезагружать reference политику. 
	Добавление новых классов необходимо перейти в папку со скаченными исходниками
	политики. Все классы описаны в файле policy/flask/security\_classes. Добавим в
	нее следующие строчки, для описания объекта, связанного с элементом графического
	интерфейса - кнопкой:
	\begin{verbatim}
	class mybutton
	\end{verbatim}
	После добавления класса необходимо так же добавить список разрешений, связанных
	с данным классом. Для этого необходимо отредактировать файл
	policy/flask/access\_vectors. Этот файл содержит объявления общих разрешених и
	их связи с классами, а также объявления классовых разрешений. Добавим следующие
	строчки:
	\begin{verbatim}
	common gui 
	{
	    set_enabled
	    set_disabled
	}
	class mybutton
	inherits gui
	{
        click
	}
	\end{verbatim}	
	Таким образом для класса кнопки объявлены три разрешения, позволяющие
	устанавливать состояние кнопки, а так же проверять разрешение на нажатие
	кнопки.
	
	В процессе работы было выяснено, что reference политика не содержит класса
	service, в отличие от политики для Fedora, поэтому его понадобилось так же
	объявить в этих файлах. Без данного класса политика не работала.
	
	Параметры компиляции можно отредактировать в файле build.xml.
	Я изменял два параметра, остальные параметры оставил по умолчанию:
	\begin{verbatim}
	#Тип политики
	TYPE=mls
	# Тип дистрибутива. Для Fedora необходимо выбирать redhat
	DISTRO=redhat
	\end{verbatim}
	Перед компиляцией необходимо произвести конфигурацию политики. Выполняется это
	с помощью команды:
	\begin{verbatim}
	make conf
	\end{verbatim}
	Это команда создаст файл с описанием модулей, включаемых в политику, при
	желании его можно отредактировать.
	
	Все готово к компиляции, выполняем команду:
	\begin{verbatim}
	make
	\end{verbatim}
	Компиляция может занять некоторые время, по завершению будут созданы бинарные
	файлы для политики и отдельных ее модулей. Далее необходимо установить
	скомпилированную политику и загрузить ее (выполнять данные действия необходимо
	от суперпользователя):
	\begin{verbatim}
	make install
	make load
	\end{verbatim}
	По умолчанию, скомпилированная политика будет загружаться в директорию
	/etc/selinux/refpolicy. При необходимости директорию refpolicy можно сменить на
	другую, это делается в файле build.xml перед компиляцией. 
	
	Политика установлена, однако для того, что бы при включении загружалась именно
	эта политика, необходимо ее прописать в конфигурационном файле
	/etc/selinux/config:
	\begin{verbatim}
	SELINUXTYPE=refpolicy
	\end{verbatim}
	После перезагрузки Linux должен загрузить скомплированную политику. Проверить
	это можно отобразив список всех классов текущей политики, среди него должен
	быть добавленный класс mybutton:
	\begin{verbatim}
	[vitalan69@UIR8 ~]$ seinfo -cmybutton -x
   	mybutton
       set_disabled
       set_enabled
       click
	[vitalan69@UIR8 ~]$ 
	\end{verbatim}
	После этого данный класс и его разрешения можно использовать в модулях
	политики.
	
	\subsection{Написание модуля полити безопастности}
	Подгружаемый модуль политики безопастности состоит из трех файлов с
	расширениями .if, .te, .fc. Названия файлов должны совпадать с названием
	модуля. Назовем модуль для описания правил работы с классом кнопки	mybutton. 
	Файл mybutton.te содержит объявления типов и основные правила вектора доступа и
	перехода типов. Файл имеет следующее содержание:
	\begin{verbatim}
	#Первая строка объявляет имя и версию модуля
	policy_module(mybutton,1.0)

#Макрос, позволяющий использовать в тексте модуля соответствующие классы,
#разрешение, типы и роли
gen_require(` 
    class mybutton click;
    type user_t;
    type sysadm_t;
    role user_r;	
    role sysadm_r;
    type chkpwd_t;
')

# Объявления типов трех кнопок
type mybutton_red_t;
type mybutton_yellow_t;
type mybutton_green_t;


# Объявление типов домена для администратора и пользователя, 
#в которых будет рабоать программа с кнопками
type mybutton_adm_domain_t;
type mybutton_user_domain_t;

# Объявление типа выполняемого файла программы
type mybutton_exec_t;

# Данный макрос указывает, что добавленные типы являются именно доменами
domain_type(mybutton_user_domain_t)
domain_type(mybutton_adm_domain_t)

# Связывание ролей пользователя и администратора 
# с добавленными нами доменами
role user_r types {mybutton_user_domain_t};
role sysadm_r types {mybutton_adm_domain_t};

# Правила перехода типов домена. 
type_transition user_t mybutton_exec_t:process mybutton_user_domain_t; 
type_transition sysadm_t mybutton_exec_t:process mybutton_adm_domain_t;
type_transition chkpwd_t mybutton_exec_t:process mybutton_adm_domain_t;

#Разрешения доступа для пользователя к двум типам кнопок
allow mybutton_user_domain_t mybutton_yellow_t:mybutton click;
allow mybutton_user_domain_t mybutton_green_t:mybutton click;

#Разрешения доступа для администратора к двум типам кнопок
allow mybutton_adm_domain_t mybutton_red_t:mybutton  click;
allow mybutton_adm_domain_t mybutton_green_t:mybutton click;

\end{verbatim}
	Данный модуль политики указывает, что бы исполняемый файл, имеющий тип
	mybutton\_exec\_t запускался в разных доменах для пользователя и
	администратора. Так же в данной политике для администратора разрешен доступ к
	красной и зеленой кнопке, а для пользователя к желтой и зеленой кнопке. 
	
	Еще в модуле политики указывается, что роль пользователя имеет право выполнять
	домен mybutton\_user\_domain\_t, а роль администратора имеет право выполнять
	домен mybutton\_adm\_domain\_t. Без этого переход типов будет осуществлять
	некорректно.
	
	Файл mybutton.if содержит интерфейсы модуля. В рамках поставленной задачи не
	требуется предоставлять интерфейс к модулю. Однако в этом файле можно добавить
	документацию к модулю.
	\begin{verbatim}
## <summary>Mybutton exemple policy</summary>
## <desc>
##	<p>
##		Provide different access for users and sysadm to 
##      objects class mybutton.
##	</p>
## </desc>

#################################
	\end{verbatim}
	
	Файл mybutton.fc содержит правила для маркирования файлов. Нам необходимо
	указать контекст для исполняемого файла программы, демонстрирующей возможности
	добавленного класса. Содержащие файла:
	\begin{verbatim}
/home/vitalan69/workspace/uir/uir  -- gen_context(user_u:user_r:mybutton_exec_t,s0)
	\end{verbatim}
	
	Можно приступать к компиляции модуля. Для компиляции используется Makefile,
	расположенный в директории /usr/share/selinux/devel/. Выполняем команду:
	\begin{verbatim}
	make -f /usr/share/selinux/devel/Makefile
	\end{verbatim}
	По завершению компиляции должен появиться бинарный файл модуля политики с
	расшерением .pp
	
	Установка модуля политики производится с помощью утилиты semodule:
	\begin{verbatim}
	semodule -i mybutton.pp
	\end{verbatim}
	
	Проверить, загружен ли правила из модуля в систему можно с помощью утилиты
	apol, которая позволяет отображать список всех правил, а так же производить
	поиск по ним. 
	
	 \subsection{Программная проверка доступа с помощью библиотеки
	libselinux}
	Для демонстрации результатов написанной политики была написана программа с
	пользовательским интерфейсом. Для разработки пользвотельского интерфейса
	использовался QT фрэймворк. Программа имеет следующий вид:
	 \begin{figure}
	\image[width=50mm]{./../img/program.png}
	\Description{Интерфейс программы}
     \caption{Интерфейс программы}
     \end{figure}
     Если доступ разрешен пользователю показывается сообщение с текстом
     Access allowed, иначе показывается сообщение Access deny.
     
     Программа используется библиотеку libselinux для проверки доступа, поэтому
     ее надо подключить в профайле проекта:
     
     \begin{verbatim}
LIBS += -L/usr/lib/ -lselinux
     \end{verbatim}
     
     Перед проверкой доступа надо получить контекст текущего процесса, а так же
     активировать работу с вектором доступа:
     \begin{verbatim}
void MainWindow::initSelinux()
{
    getcon(&program_context);
    qDebug()<<"Program context = " << program_context;
    selinux_opt *opt = new selinux_opt();
    opt->type = 1;
    opt->value = "avc";
    avc_open(opt, AVC_OPT_SETENFORCE);
}
     \end{verbatim}
    В программе для каждой из кнопок заданны контексты с соотвествующим типом.
    При клике на кнопку проверяется доступ с помощью следующей функции:
    \begin{verbatim}
int MainWindow::checkAccess(security_context_t buttoncon){
    security_id_t myid = NULL;
    struct av_decision avd = {0, 0, 0, 0, 0,0};
    security_id_t buttonid = NULL;
    avc_context_to_sid(program_context, &myid);
    avc_context_to_sid(buttoncon, &buttonid);
    access_vector_t av=0;
    int length=security_compute_av(program_context,buttoncon,SECCLASS_MYBUTTON,av,&avd);
    int result = avc_has_perm(myid, buttonid, SECCLASS_MYBUTTON, MYBUTTON__CLICK , NULL, &avd);
    QMessageBox msgBox;
    if (result!=0&&(errno==EACCES)){
        qDebug()<<"Errno="<<errno;
        msgBox.setText(tr("Access deny"));
    } else if (result==0){
        msgBox.setText(tr("Access allowed"));
    } else {
        msgBox.setText(tr("Error "));
        qDebug()<<"errno="<<errno;
    }
    msgBox.exec();
    qDebug()<<"Length="<<length;
    qDebug()<<avd.allowed<<avd.auditallow<<avd.auditdeny<<avd.decided<<avd.flags<<avd.seqno;
    return result;
}
    \end{verbatim}
	Наиболее интересна в данной функции строчка:
	\begin{verbatim}
int result = avc_has_perm(myid, buttonid, SECCLASS_MYBUTTON, MYBUTTON__CLICK ,	NULL, &avd);
	\end{verbatim}
	В ней проверяется доступ процесса с SID myid к объекту с SIC buttonid,
	указывается что объект buttonid принадлежит к классу SECCLASS\_MYBUTTON, а
	разрешение, которое запрашивается для него - MYBUTTON\_\_CLICK.
	В результате, если возвращается 0, то доступ разрешен. Если возвращается -1, то
	это означает, что либо доступ запрещен, либо произошла другая ошибка. Если
	доступ запрещен, то то значение errno устанавливается в значение EACCES.
	
	Для демонстрации работы программы необходимы два пользователя, первый работает
	с ролью user\_r, второй с ролью sysadm\_r. Пользователю user\_r разрешен доступ
	к желтой и зеленой кнопкам, а пользователю sysadm\_r разрешен доступ к красной
	и зеленой кнопкам. 
	
	
% Заключение   
\StructureElement{Заключение}
 В процессе выполнения задания были рассмотрены ключевые особенности мандатного
 разграничения доступа, использующегося в SELinux. Был изучен и применен на
 практике язык написания политик для модуля безопастности SELinux. Результатом
 работы можно считать описание алгоритма добавления новых классов доступа в
 политику SELinux, а так же проверки доступа с помощью библиотеку libselinux. 
 
 Результат работы можно использовать для написания приложений с пользовательским
 интерфейсом, в котором необходимо разграничить доступ пользователя к различным
 элементам управления в зависимости от уровня доступа пользователя.
 
 Кроме того, был получен опыт написания и редактирования существующих политик
 SELinux, который можно использовать при настройки и администрирование
 компьютеров с ОС Linux, на которых включен модуль безопастности SELinux.
 
% Отзыв руководителя (шаблон для рукописного заполнения)
\ReviewOfSupervisorTemplate

% Отзыв руководителя
\begin{ReviewOfSupervisor}

\end{ReviewOfSupervisor}

% Задание на УИР (шаблон для рукописного заполнения)
\TaskOfStudentTemplate

% Задание на УИР
\begin{TaskOfStudent}
В рамках выполнения работы необходимо: 
\begin{enumerate}
  \item Изучить основы работы с политиками SELinux
  \item Добавить новый класс доступа в политику SELinux
  \item Написать модуль политики безопастности для добавленного класса
  \item Реализовать проверку доступа, используя библиотеку libselinux для
  добавленного класса
\end{enumerate}
\end{TaskOfStudent}



\end{document}
